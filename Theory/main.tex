\graphicspath{{Theory/}}
This chapter presents the background information that this thesis is built on. An overview of GNSS is given, followed by a look into the integration of INS and GNSS and ending with a quick overview of the major software packages used in the implemented system. 

\section{Global Navigation Satellite Systems}
    \todo{reference prosjektoppgave}
    \subimport{Theory/}{gnssNew.tex}

\section{Integrating INS and GNSS}
    \subimport{Theory/}{integration.tex}

\section{Software packages}
    \subimport{Theory/}{software-packages.tex}
%Structure
    % 1 - INS
    % 2 - GNSS
    % 3 - Integration
    
%x Theory
%x  - GNSS (raskt renskrevet fra prosjektoppgave)
%x  - Integration
%x     1 - Motivation
%x         + Improving state estimation
%x     2 - Architectures
%x         + Loose
%x         + Tight
%x         + Ultra tight
%x         + Feedback vs open loop
%x     3 - Kalman filter
%x         + KF / EKF
%x           - Indirect implementation
%x               - Orientations state minimal. No over-parametrization
%x               - Operating close to the origin. Far away from singularities
%x               - Always small. Second-order products negligible
%x               - Error dynamics are small. All large dynamics already in nominal state. Can apply
%x               - corrections at a lower rate than predictions 
%x               - \cite{sola2017quaternion}
%x           - MEKF
%x               - Maintains quaternion unit length
%x     4 - Parametrizations
%x         + Vector-angle
%x         + Euler
%x         + Quaternion (Linear diff. eq. No EKF divergence)
%x         + Gibbs
%x         + Modified Rodrigues
    