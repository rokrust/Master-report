\todo{Use a table?}
\todo{Write things specifically for my report? Body frame orientation and so on}
\todo{Add to abbrev}

\section{ECI}
    The Earth-centered inertial (ECI) frame has its origin at the center of the Earth and z-axis intersecting the geographic North Pole. It is denoted by \{i\}.
    

\section{ECEF}
    The Earth-centered, Earth-fixed (ECEF) frame is another coordinate system with origin at the center of mass of the Earth. Its z-axis is also shared with the ECI reference frame, but the x- and y-axes intersects fixed point on the surface of the earth, meaning that it rotates with the Earth about the z-axis. As such, ECEF is not an inertial frame, but for slow moving craft it may be considered as such \cite{fossen2011handbook}. In this thesis, the ECEF frame of reference is denoted by \{e\}. 
    
    \begin{figure}
        \centering
        \includegraphics[scale=1]{Appendices/bilder/ECEF_ENU_BETTER.png}
        \caption{NED and ECEF frames}
        \label{fig:ecef-enu}
        \todo{find better figure?}
    \end{figure}
    
\section{NED}
    The north-east-down frame is a location dependent system tangential to the surface of the Earth. The z-axis points down, along the normal of the Earth ellipsoid, while x- and y-axes points north and east, respectively. The NED frame is denoted \{n\}.

\section{Body}
    The body-fixed reference frame is a moving coordinate system fixed to some craft. Both origin and orientation of the coordinate system is chosen freely \todo{And has in this thesis been set to..}. The body system is denoted \{b\}

\section{Transformations between frames}
    \subsection{NED and ECEF}
        \label{app:roten}
        According to \cite{fossen2011handbook}, the rotation matrix from NED to ECEF is given by 
        \begin{equation}
            \label{eq:Ren}
            R^e_n =
            \begin{bmatrix}
                -cos(l)sin(\mu) & -sin(l)   & -cos(l)cos(\mu)\\
                -sin(l)sin(\mu) & cos(l)    & -sin(l)cos(\mu)\\
                cos(\mu)       & 0          & -sin(\mu)\\
            \end{bmatrix}
        \end{equation}
        
        where $l$ denotes latitude and $\mu$ denotes longitude.
    
    \subsection{Body and NED}
        \label{app:rotnb}
        Given a set of Euler angles $\begin{bmatrix}\phi & \theta & \psi \end{bmatrix}$ denoting the rotation from body to NED, a rotation matrix can be calculated as
        \begin{equation}
            \label{eq:Rnb-eul}
            \mathbf{R}^n_b =
            \begin{bmatrix}
                \cps\cth & -\sps\cph+\cps\sth\sph   & \sps\sph+\cps\cph\sth\\
                \sps\cth & \cps\cph+\sph\sth\sps    & -\cps\sph+\sth\sps\cph\\
                -\sth    & \cth\sph                 & \cth\cph\\
            \end{bmatrix}
        \end{equation}
        Where $c$ and $s$ is shorthand for the cosine and sine functions, respectively. Alternatively, the rotation matrix can be calculated from the quaternion $\begin{bmatrix}\eta & \varepsilon_1 & \varepsilon_2 & \varepsilon_3 \end{bmatrix}$ as
        \begin{equation}
            \label{eq:Rnb-quat}
            \mathbf{R}^n_b =
            \begin{bmatrix}
                1-2(\varepsilon_2^2+\varepsilon_3^2)   &2(\varepsilon_1\varepsilon_2-\varepsilon_3\eta) &2(\varepsilon_1\varepsilon_3+\varepsilon_2\eta)\\
                2(\varepsilon_1\varepsilon_2+\varepsilon_3\eta)      &1-2(\varepsilon_1^2+\varepsilon_3^2)   &2(\varepsilon_2\varepsilon_3-\varepsilon_1\eta)\\
                2(\varepsilon_1\varepsilon_3-\varepsilon_2\eta)      &2(\varepsilon_2\varepsilon_3+\varepsilon_1\eta) &1-2(\varepsilon_1^2+\varepsilon_2^2)\\
            \end{bmatrix}
        \end{equation}