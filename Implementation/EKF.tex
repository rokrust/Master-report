The chosen integration filter of this thesis is a simple direct state extended Kalman filter. Before the implementation is presented, the EKF equations are shown.
\subsection{Equations} \todo{Rename}
    The discrete time extended Kalman filter method can be divided into a prediction step, based on the state model, and a measurement step as follows.
    \paragraph{Prediction}
    \todo{Fix equations}
    \begin{align}
        \intertext{The next state estimate is predicted based on the system model}
        \xk^- &= \bm{f(\xkp, \atkp{u})}\\
        \intertext{The error covariance is propagated a priori based on the linearized model and process noise covariance}
        \atk{P}^- &= \atk{\Phi}\atkp{P}\atk{\Phi}^T + \atk{Q}
    \end{align}
    
    \paragraph{Measurement}
    \begin{align}
        \intertext{The measurement step estimates an incoming set of measurements and calculates a measurement residual, $\atk{z}$, and its covariance, $\atk{S}$}
        \atk{z} &= \atk{y} - \bm{h}(\xk^-)\\
        \atk{S} &= \atk{H}\atk{P}^-\atk{H}^T+\atk{R}\\
        \intertext{The posterior state estimate, $\xk$ is calculated based on the average of the priori state and measurement residual weighted by the Kalman gain $\atk{K}$}
        \atk{K} &= \atk{P}^-\atk{H}^T\atk{S}^{-1}\\
        \xk     &= \xk^- + \atk{K}\atk{z}\\
        \intertext{The error covariance is then updated. This form of the update is known as the Joseph form and avoids potential problems with numerical stability}
        \atk{P} &= (\eye{n} - \atk{K}\atk{H})\atk{P}^-(\eye{n} - \atk{K}\atk{H})^T + \atk{K}\atk{R}\atk{K}^T
    \end{align}
    
\subsection{Model description}
    %Describe state 
    The implemented model estimates position and velocity in the ECEF frame of reference, accelerometer bias and receiver clock bias and receiver clock bias rate. Note that all GNSS constellations keep a different time system, and it is therefore necessary to add additional bias estimates when working with a multi-GNSS setup. This thesis works with measurements from both GPS and GLONASS, and hence the offset bias between the two is estimated as well. In that order, this leads to the state space vector
    \begin{equation}
        \bm{x} = 
        \begin{bmatrix}
            \pos\\[6pt]
            \vel\\[6pt]
            \ba\\[6pt]
            \bias\\[6pt]
            \biasdot\\[6pt]
            \biasglo
        \end{bmatrix}
    \end{equation}
    
    Note that the attitude is left unmodeled in the filter, although the prediction model depends on the rotation matrix from the body frame to the ECEF frame. Therefore, the attitude is modeled separately by a different system and assumed to be a perfect representation. It also depends on the gravitational vector decomposed in the ECEF frame. This is found by rotating the \textit{plumb bob} gravity vector, defined in the NED frame as

    \begin{equation}
        \label{def:gn}
        \mathbf{g}_n =
        \begin{bmatrix}
            0 & 0 & 9.80665
        \end{bmatrix}^T
        [\frac{m}{s^2}]
    \end{equation}
    
\todo{Position and velocity in ECEF}

\subsubsection{Measurement model}
    The EKF measurement vector is 
    \begin{equation}
        y = \begin{bmatrix}P & \dot{P}\end{bmatrix}^T    
        \label{eq:measurement}
    \end{equation}
    composed of the pseudo-range and pseudo-range rate respectively. The measurement model is based on equations \ref{eq:pr-model} and \ref{eq:prr-model}. However, all corrections, except receiver clock bias, are applied outside of the Kalman filter, and are therefore not included in the EKF measurement model. This yields the model
    \begin{align}
        \pr       &= \range + \beta + \vartheta\\ \todo{fix notation}
        \dot{\pr} &= \dot{\range} + \dot{\beta} + \dot{\vartheta}
    \end{align}
    
    The linearized measurement matrix can be split into a pseudo-range and a pseudo-range rate part as follows
    \begin{equation}
        \bm{H} = \begin{bmatrix} \bm{H}_{\pr}\\\bm{H}_{\dot{\pr}}\end{bmatrix}
    \end{equation}
    Both will now be derived.    
    
    \subsubsection*{Pseudo-range}
    To derive the linearized measurement matrix for the pseudo-range measurements, $\bm{H}_\pr$, the model is first differentiated with respect to the receiver position
    \begin{align}
        \begin{split}
            \partialdiff{\pr}{\pos}   & = \partialdiff{}{\pos}\norm{\bm{p}^e_s - \pos}_2\\   & = -\Bigg(\frac{\bm{p}^e_s - \pos}{\norm{\bm{p}^e_s - \pos}_2}\Bigg)^T = \bm{l}^{e^T}
        \end{split}
        \\[2ex]
        \intertext{Where $\bm{l}^e^T$ is the line of sight vector between receiver and satellite. Differentiating with respect to the receiver clock bias and GLONASS offset is trivial}
        \begin{split}
            \partialdiff{\pr}{\bias}   & = 1\\
            \partialdiff{\pr}{\biasglo}   & = 1
        \end{split}
    \end{align}
    
    The model does not depend on any other state variables. Given $m$ distinct pseudo-range measurements the matrix form becomes
    \begin{equation}
        \bm{H}_\pr =
        \begin{bmatrix}
            -\bm{l}_1^{e^T} & \zeros{1}{3} & \zeros{1}{3} & 1 & 0 & 0\\
            -\bm{l}_2^{e^T} & \zeros{1}{3} & \zeros{1}{3} & 1 & 0 & 0\\
            &\vdots\\
            -\bm{l}_m^{e^T} & \zeros{1}{3} & \zeros{1}{3} & 1 & 0 & 0\\
        \end{bmatrix}
    \end{equation}
    
    Where measurements from the GLONASS satellites will add a one to the last column to include the offset estimate.
    
    \subsubsection*{Pseudo-range rate}
    Similarly, the pseudo-range rate model is linearized too. From equation \ref{eq:range-rate} it is seen that it depends on both position, velocity and receiver clock bias rate. However, because of the Doppler shift's weak dependence on position, only the velocity and receiver clock bias rate terms are used.
    
    \begin{align}
        \begin{split}
            \partialdiff{\dot{\pr}}{\vel}   & = \partialdiff{}{\vel}(\bm{v}^e_s - \vel)\bm{l}^{e^T} = -\bm{l}^{e^T}
        \end{split}
        \\[2ex]
        \begin{split}
            \partialdiff{\pr}{\biasdot}   & = 1
        \end{split}
    \end{align}
    
    The measurement matrix for GPS satellites are thus
    \begin{equation}
        \bm{H}_\pr =
        \begin{bmatrix}
            \zeros{1}{3} & -\bm{l}_1^{e^T} & \zeros{1}{3} & 0 & 1 & 0\\
            \zeros{1}{3} & -\bm{l}_2^{e^T} & \zeros{1}{3} & 0 & 1 & 0\\
            &\vdots\\
            \zeros{1}{3} & -\bm{l}_m^{e^T} & \zeros{1}{3} & 0 & 1 & 0\\
        \end{bmatrix}
    \end{equation}

\subsubsection{Prediction model}
    The acceleration measurement, $\bm{f}^b_{IMU}$, is the input vector of the system and is modeled in the body frame as 
    \begin{equation}
        \begin{split}
            \bm{f}^b_{IMU} = \bm{f}^b_{ib} + \ba + \bm{w}_a\\
            \bm{f}^b_{ib} = \bm{f}^b_{IMU} - \ba - \bm{w}_a
        \end{split}
        \label{eq:acc-mes-model}
    \end{equation}
    where $\bm{w}_a$ is considered Gaussian white noise.
    
    \begin{align}
        \intertext{The time derivative of position is simply defined as the velocity}
        \posd &= \vel\\
        \label{eq:pred-vel}
        \intertext{Rotating the acceleration measurement \ref{eq:acc-mes-model} and adding a Coriolis term to account for earth rotation, the time derivative of the velocity can be written}
        \veld &= \rot{e}{b}(\fimu - \ba) + \bm{g}^e -2\bm{S}(\omgie)\vel\\\todo{fix notation}
        \label{eq:pred-pos}
        \intertext{Where the acceleration measurement noise is discarded as it has expected value 0 and the rotation matrix $\rot{e}{b}$ is estimated outside of the filter. The GPS receiver clock bias and its rate is trivial. The GLONASS offset is assumed constant in the prediction model}%Same drift rate
        \biasd &= \biasdot\\
        \biasdotd &= 0\\
        \biasglod &= 0
    \end{align}
    
    %Linearized state matrix
    As the filter does not estimate the attitude, this prediction model can be written on state space form
    \begin{align}
        \dot{\bm{x}} &= \bm{F}\bm{x} + \bm{B}\bm{u}^b_{ib}
        \label{eq:state-space}
        \intertext{Where $\bm{x}$ is the state vector and $\bm{u}$ is the acceleration measurement vector. The transition and input matrix is}
        \bm{F} &= 
        \begin{bmatrix}
            \zeros{3}{3} & \eye{3}          & \zeros{3}{3}\\
            \zeros{3}{3} & -2\bm{S}(\omgie) & \rot{e}{b}  \\
            \zeros{3}{3} & \zeros{1}{3}     & \zeros{3}{3}\\
            \zeros{1}{3} & \zeros{1}{3}     & \zeros{1}{3}\\
            \zeros{1}{3} & \zeros{1}{3}     & \zeros{1}{3}\\
        \end{bmatrix}\\
        \bm{B} &= 
        \begin{bmatrix}
            \zeros{3}{3}\\
            \rot{e}{b}\\
            \zeros{3}{3}\\
            \zeros{1}{3}\\
            \zeros{1}{3}\\
            \zeros{1}{3}
        \end{bmatrix}
    \end{align}
    Where the rotation matrix, $\rot{e}{b}$, is calculated from the attitude received from a separate system.
    
\subsubsection{Discretization}
    Because the filter is implemented on a computer, the dynamics should be discretized. Defining the acceleration vector
    \begin{align}
        \bm{a} &:= \rot{e}{b}(\fimu - \ba) + \bm{g}^e - 2\bm{S}(\omgie)\eb{\hat{v}}\\
        \intertext{the prediction model becomes}
        \eb{\hat{p}} &\leftarrow \eb{\hat{p}} + \eb{\hat{v}}\Delta t + \bm{a}\frac{\Delta t^2}{2}\\
        \eb{\hat{v}} &\leftarrow \eb{\hat{v}} + \bm{a}\Delta t
    \end{align}
    
    Discretization of the process noise covariance matrix normally requires a rather involved integral to be solved. However, the Van Loan method described by algorithm \ref{alg:vanloan} can be used to find both the discretized state transition matrix and process noise covariance matrices \cite{van1978computing}

    \begin{algorithm}[!htbp]
        \caption{Van Loan's discretization method}
	    \label{alg:vanloan}
        \begin{minipage}{\dimexpr\textwidth-2\algomargin\relax}
            \begin{enumerate}
            \item First, form the $2n\times2n$ matrix $\bm{M}$:
            \begin{equation}
                \mathbf{M} := \left[\begin{array}{cc}-\mathbf{F} & {\mathbf{Q}} \\ {\mathbf{0}} & {\mathbf{F}^{\top}}\end{array}\right]
            \end{equation}
            \item Taking the matrix exponential of $\bm{M}$ yields the following relationship to the discretized system
            \begin{equation}
                e^{\mathbf{M} \Delta t} = \left[ \begin{array}{cc}{\cdots} & {\bm{\Phi}_{k}^{-1} \mathbf{Q}_{k}} \\ {0} & {\bm{\Phi}_{k}^{\top}}\end{array}\right]
            \end{equation}
            \item $\bm{Q}_k$ is then found from the upper right matrix as
            \begin{equation}
                \atk{Q} = \atk{\Phi}\atk{\Phi}^{-1}\atk{Q}
            \end{equation}
            \end{enumerate}
            \vspace{0.1cm}
        \end{minipage}
    \end{algorithm}

% R = a*(1/sin(el) + c)
% Faster convergence e^-ax + e^-bx + c
%Structure
    % 1 - EKF method description
    %     + Because the expected error state is always zero except between a measurement correction          and the injection of the error in to the nominal state, followed by a reset, there is no         point in actually implementing the propagation of the error state. The covariance of the         error is however propagated (Ingen x priori)
    % 2 - Paremetrization description and state vector forms
    %     + Quaternions have 3 DOF, but four parameters. P might be rank deficient \cite{trawny2005indirect, sola2017}
    %      + Gibbs vector has a singularity at 180\deg. 
    % 3 - Model description (matrices and functions)
          
    %     + Discretization \cite{van1978computing, wahlstrom2014discretizing}:
    % 4 - Observability
    % 5 - Noise models
    %     + Eksponentielt synkende støymodell (V1*e^-T1*t + V2*e^-T2*t + V3)
    % 6 - Challenges
    %     + Discretization timestep
    
%but over-paramatrization can make the Kalman covariance matrix rank deficient